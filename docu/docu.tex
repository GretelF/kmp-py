\documentclass[12pt,a4paper]{scrartcl}
\usepackage[utf8]{inputenc}
\usepackage{graphicx}
\usepackage{url}
\usepackage{amsmath}
\usepackage{caption}
\usepackage{wrapfig}
\usepackage{eurosym}
\usepackage{biblatex}
\usepackage{url}
\usepackage{color}
\usepackage{listings}
\usepackage{hyperref}
\usepackage[table]{xcolor}
\linespread{1.4}

\definecolor{mygreen}{rgb}{0,0.6,0}
\definecolor{mygray}{rgb}{0.5,0.5,0.5}
\definecolor{mylightgray}{rgb}{0.7,0.7,0.7}
\definecolor{mylightergray}{rgb}{0.9,0.9,0.9}
\definecolor{mymauve}{rgb}{0.58,0,0.82}

\let\origitemize\itemize
\def\itemize{\origitemize\itemsep0pt}

\lstset{ 
  backgroundcolor=\color{white},   
  basicstyle=\ttfamily\footnotesize,          
  breakatwhitespace=false,         
  breaklines=true,  
  commentstyle=\color{mygreen}, 
  escapeinside={\%*}{*)}, 
  extendedchars=true,             
  keepspaces=true,                 
  keywordstyle=\color{blue},
  language=Octave,
  numbers=left,                   
  numbersep=15pt,                  
  numberstyle=\tiny\color{mygray}, 
  showspaces=false,                
  showstringspaces=false,          
  showtabs=false,                  
  stringstyle=\color{mymauve},
  tabsize=2,
  title=\lstname,
  captionpos=b
}

\renewcommand*\lstlistingname{Codebeispiel}    %Rename Listings

\renewcommand*\thesection{\arabic{section}}

\makeatletter
\renewcommand\subparagraph{\@startsection{subparagraph}{5}{\parindent}%
    {3.25ex \@plus1ex \@minus .2ex}%
    {0.75ex plus 0.1ex}% space after heading
    {\normalfont\normalsize\bfseries}}
\makeatother

\begin{document}
\title{Lisp Interpreter in Python 3.4}
\subtitle{Concepts of Modern Programming Languages}
\author{Maria Florus\ss}
\maketitle
\newpage

\section*{Overall Structure}

\subsection*{Lisp Objects}
\subsection*{TODO}

\section*{Functionality}

\section*{Builtin Syntax}
\subsubsection*{define}
\subsubsection*{lambda}
\subsubsection*{if}
\subsubsection*{set!}
\subsubsection*{let}
\subsubsection*{begin}
\subsubsection*{quote}
\subsubsection*{and}
\subsubsection*{or}
\section*{Builtin Functions}

\subsection*{Arithmetic}

\subsubsection*{add}
\begin{tabular}{l  p{13cm}}
Description: & Adds an arbitrary amount of numbers and returns the accumulated value as \lstinline{SchemeNumber}. If only one argument is given, the arguments value is returned as \lstinline{SchemeNumber}. If no argument is given the return value is 0.\\
Symbol: & \lstinline{+}\\
Arguments: & 0+ \lstinline{SchemeNumber}s\\
\end{tabular}
\\
\\
Example of usage:
\begin{lstlisting}
> (+ 1 2)
3
> (+ 2 3 4)
9
> (+)
0
> (+ 42)
42
\end{lstlisting}

\subsubsection*{subtract}
\begin{tabular}{l  p{13cm}}
Description: & Subtracts an arbitrary amount of numbers from the first number and returns the accumulated value as \lstinline{SchemeNumber}. If only one argument is given, the arguments value is negated and returned as \lstinline{SchemeNumber}. If no argument is given an ArgumentCountException is risen.\\
Symbol: & \lstinline{-}\\
Arguments: & 1+ \lstinline{SchemeNumber}s\\
\end{tabular}
\\
\\
Example of usage:
\begin{lstlisting}
> (- 0.5 2)
-1.5
> (- 10 3 4)
3
> (-)
ArgumentCountException: 'function - expects at least 1 argument.'                                                        
> (- 42)
-42
\end{lstlisting}

\subsubsection*{multiply}
\begin{tabular}{l  p{13cm}}
Description: & Multiplies an arbitrary amount of numbers and returns the resulting value as \lstinline{SchemeNumber}. If only one argument is given, the arguments value is returned as \lstinline{SchemeNumber}. If no argument is given the return value is 1.\\
Symbol: & \lstinline{*}\\
Arguments: & 0+ \lstinline{SchemeNumber}s\\
\end{tabular}
\\
\\
Example of usage:
\begin{lstlisting}
> (* 3.5 4)
14.0
> (* 2 3 4)
24
>(*)
1
> (* 42)
42
\end{lstlisting}

\subsubsection*{divide}
\begin{tabular}{l  p{13cm}}
Description: & Divides the first argument by the second, the result by the third and so on. If only one argument is given, the result is 1 devided by the argument. If no argument is given an ArgumentCountException is risen.\\
Symbol: & \lstinline{/}\\
Arguments: & 1+ \lstinline{SchemeNumber}s\\
\end{tabular}
\\
\\
Example of usage:
\begin{lstlisting}
> (- 0.5 2)
-1.5
> (/ 12 3 2)
2.0
> (/)
ArgumentCountException: 'function - expects at least 1 argument.'                                                        
> (/ 3)
0.3333333333333333
\end{lstlisting}
\subsubsection*{arithmetic equals}
\subsubsection*{modulo}

\end{document}