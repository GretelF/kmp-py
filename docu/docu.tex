
\documentclass[12pt,a4paper]{scrartcl}
\usepackage[utf8]{inputenc}
\usepackage{graphicx}
\usepackage{url}
\usepackage{amsmath}
\usepackage{caption}
\usepackage{wrapfig}
\usepackage{eurosym}
\usepackage{biblatex}
\usepackage[english]{babel}
\usepackage{url}
\usepackage{color}
\usepackage{listings}
\usepackage{hyperref}
\usepackage[table]{xcolor}
\linespread{1.4}
\usepackage{todonotes}

\definecolor{mygreen}{rgb}{0,0.6,0}
\definecolor{mygray}{rgb}{0.5,0.5,0.5}
\definecolor{mylightgray}{rgb}{0.7,0.7,0.7}
\definecolor{mylightergray}{rgb}{0.9,0.9,0.9}
\definecolor{mymauve}{rgb}{0.58,0,0.82}

\let\origitemize\itemize
\def\itemize{\origitemize\itemsep0pt}

\lstset{ 
  backgroundcolor=\color{white},   
  basicstyle=\ttfamily\footnotesize,          
  breakatwhitespace=false,         
  breaklines=true,  
  commentstyle=\color{mygreen}, 
  escapeinside={\%*}{*)}, 
  extendedchars=true,             
  keepspaces=true,                 
  keywordstyle=\color{blue},
  language=Octave,
  numbers=left,                   
  numbersep=15pt,                  
  numberstyle=\tiny\color{mygray}, 
  showspaces=false,                
  showstringspaces=false,          
  showtabs=false,                  
  stringstyle=\color{mymauve},
  tabsize=2,
  title=\lstname,
  captionpos=b
}

\renewcommand*\lstlistingname{Codebeispiel}    %Rename Listings

\renewcommand*\thesection{\arabic{section}}

\makeatletter
\renewcommand\subparagraph{\@startsection{subparagraph}{5}{\parindent}%
    {3.25ex \@plus1ex \@minus .2ex}%
    {0.75ex plus 0.1ex}% space after heading
    {\normalfont\normalsize\bfseries}}
\makeatother

\begin{document}
\title{Lisp Interpreter in Python 3.4}
\subtitle{Concepts of Modern Programming Languages}
\author{Maria Floru\ss}
\maketitle
\newpage

\section*{General Structure}
\begin{itemize}
\item This interpreter is recursive and does not implement continuations. 
\item No self-written garbadge collector is implemented. The python GC is currently cleaning.
\item All needed objects where implemented as lisp objects (see below for more information).
\end{itemize}

\todo{more}

\section*{Functionality}
A Read-Eval-Print-Loop (REPL) allows the user to continue interacting, even if an error occurred. An error message is shown in this case.

It is possible to structure your input in multiple lines. The REPL will count the opening and closing parethesis and will only redirect to the reader if each opening parathesis has a closing counter part. This leads to multiline input.

\todo{more}

\subsection*{Lisp Objects}
All object types, that were needed during the implementation, were implemented as \lstinline{SchemeObject}s. The base class \lstinline{SchemeObject} contains all necessary functions, such as \lstinline{isTrue()} which returns \lstinline{SchemeTrue} or a basic compare and \lstinline{toString} function. I decided to use this structure to have a possibility to use all \lstinline{SchemeObject}s in my interpreter, not just the obvious ones like \lstinline{SchemeString} and \lstinline{SchemeCons}, but also for example \lstinline{SchemeEnvironment} and \lstinline{SchemeStringStream}. 

All implemented objects extend the base class and override some or all default functionalities. \lstinline{SchemeFalse} for example overrides \lstinline{isTrue()} and returns \lstinline{SchemeFalse} instead.

I wanted some objects to implement the singleton pattern. I achieved this by creating a special object \lstinline{SchemeSingleton}, which serves as base class for the objects \lstinline{SchemeTrue}, \lstinline{SchemeFalse}, \lstinline{SchemeNil} and \lstinline{SchemeVoid}. When a new \lstinline{SchemeSingleton} is created the class will look up, if an instance already exists and will rather return this one instead of creating a new one. 

In the interpreter there are no separate objects for integer and floats. Since python is able to use arithmetic operations for both types mixed up, I implemented a \lstinline{SchemeNumber} object, which can contain both, an integer or a float. The only time where I had to diverentiate between both was the builtin function \lstinline{modulo()}, which should only work for integers. But since python provides functions like \lstinline{isinstance()}, where you can check, if an object is an instance of a certain class, this was no problem at all.

The \lstinline{SchemeEnvironment} is implemented by using pythons dictionaries. For each binding a new entry is added to the dictionary which binds a \lstinline{SchemeSymbol} to a \lstinline{SchemeObject}. Each environment can have a parent. There are two \lstinline{SchemeEnvironment}s created when the program is started: the syntax environment and the global environment where all builtinfunctions are bound. The syntax environment serves as parent for the global environment. If the system asks an environment for a special binding, it searches its own dictionary and if it can not find anything it will forward the request to it's parent environment. This way, if the evaluator asks for a binding for a syntax symbol it will first go through the global environment and if there is nothing to be found, to the syntax environment. This technically allows the user to override syntax if he wants to. However to date no possibility to implement user defined syntax is given. 



\section*{Builtin Syntax}

\subsubsection*{define}
\begin{tabular}{l  p{13cm}}
Description: & Adds a binding from the first argument to the second to the current environment.\\
Symbol: & \lstinline{define}\\
Arguments: & \lstinline{SchemeSymbol}, \lstinline{SchemeObject}\\
Return Value: & \lstinline{SchemeVoid}\\
\end{tabular}
\\
\\
Example of usage:
\begin{lstlisting}
> (define a 1)
> a
1
> (define b "hello")
> b
"hello"
\end{lstlisting}

Lambda short hand syntax:\\
\begin{tabular}{l  p{13cm}}
Description: & The lambda short hand syntax takes the first element of the first argument and uses it as name. The following elements of the first argument are the arguments of the resulting user defined function. The following arguments define the function body of the user defined function.\\
Arguments: & \lstinline{SchemeCons}, \lstinline{SchemeObject}\\
Return Value: & \lstinline{SchemeVoid}\\
\end{tabular}\\
\\
Example of usage:
\begin{lstlisting}
> (define (f n) (+ n 1))
> (f 3)
4
> (define (g n m) (print "hello") (print "world") (+ n m))
> (g 1 2)
"hello"
"world"
3
\end{lstlisting}

\subsubsection*{lambda}
\begin{tabular}{l  p{13cm}}
Description: & Creates a user defined function. The first argument is a regular list of arguments, the second is a \lstinline{SchemeCons} defining the body of the function.\\
Symbol: & \lstinline{lambda}\\
Arguments: & \lstinline{SchemeCons}, \lstinline{SchemeCons}\\
Return Value: & \lstinline{SchemeUserDefinedFunction}\\
\end{tabular}
\\
\\
Example of usage:
\begin{lstlisting}
> (define f (lambda (n m) (+ n m)))
> f
<UserDefinedFunction: f> 
> (f 2 3)
5
\end{lstlisting}


\subsubsection*{if}
\begin{tabular}{l  p{13cm}}
Description: & Checks if the condition in the first argument is true. If it is true, the second argument is evaluated, otherwise the third one is evaluated. \\
Symbol: & \lstinline{if}\\
Arguments: & Condition (everything except \lstinline{SchemeFals} evaluates to \lstinline{SchemeTrue}), Then-Part, Else-Part. \\
Return Value: & \lstinline{SchemeObject}\\
\end{tabular}
\\
\\
Example of usage:
\begin{lstlisting}
> (define a 1)
> (define b 2)
> (if (> a b) (+ a 1) (+ b 1))
3
\end{lstlisting}

\subsubsection*{set!}
\begin{tabular}{l  p{13cm}}
Description: & Checks if a binding is found for the first argument, which has to be a symbol. If the binding does not exist a exeption is risen. Else the symbol is bound to the new value. \\
Symbol: & \lstinline{set!}\\
Arguments: & \lstinline{SchemeSymbol}, \lstinline{SchemeObject} \\
Return Value: & \lstinline{SchemeVoid}\\
\end{tabular}
\\
\\
Example of usage:
\begin{lstlisting}
> (set! a 2)
> a
NoBindingException: 'No binding found for symbol a.'
> (define a 1)
> a
1
> (set! a 2)
> a
2
\end{lstlisting}

\subsubsection*{let}

\subsubsection*{begin}
\begin{tabular}{l  p{13cm}}
Description: & Evaluates one argument after another and returns the return value of the last argument. If no argument is given begin returns \lstinline{SchemeVoid}.\\
Symbol: & \lstinline{begin}\\
Arguments: & 0+ \lstinline{SchemeObject}s \\
Return Value: & \lstinline{SchemeObject} \\
\end{tabular}
\\
\\
Example of usage:
\begin{lstlisting}
> (begin (print 3) (+ 1 2) (print 4) (+ 2 3))
3
4
5
\end{lstlisting}


\subsubsection*{quote}
\begin{tabular}{l  p{13cm}}
Description: & Returns the unevaluated argument.\\
Symbol: & \lstinline{and}\\
Arguments: & \lstinline{SchemeObject}( \\
Return Value: & \lstinline{SchemeObject}\\
\end{tabular}
\\
\\
Example of usage:
\begin{lstlisting}
> (quote (+ 1 2))
(+ 1 2)
> (type? (quote (+ 1 2)))
"schemeCons"
> (quote 1 2 3)
ArgumentCountException: 'quote expects exactly 1 argument.' 
\end{lstlisting}

\subsubsection*{and}
\begin{tabular}{l  p{13cm}}
Description: & Performs a conjunction on all given arguments. Returns \lstinline{SchemeTrue} if no arguments are given. If one arg)ument is false, all following arguments are not evaluated. \\
Symbol: & \lstinline{and}\\
Arguments: & 0+ \lstinline{SchemeObject}s \\
Return Value: & \lstinline{SchemeTrue} or \lstinline{SchemeFalse}\\
\end{tabular}
\\
\\
Example of usage:
\begin{lstlisting}
> (and (+ 1 2) (> 3 2) (= 2 2))
#t
> (and (+ 1 2) (< 3 2) (= 1 1))
#f
> (and (+ 1 2) (= 1 2) (thisWouldRaiseAnErrorButDoesNotBecauseItIsNotEvaluated))
#t
\end{lstlisting}

\subsubsection*{or}
\begin{tabular}{l  p{13cm}}
Description: & Performs a disjunction on all given arguments. Returns \lstinline{SchemeFalse} if no arguments are given. If one argument is true, all following arguments are not evaluated. \\
Symbol: & \lstinline{or}\\
Arguments: & 0+ \lstinline{SchemeObject}s \\
Return Value: & \lstinline{SchemeTrue} or \lstinline{SchemeFalse}\\
\end{tabular}
\\
\\
Example of usage:
\begin{lstlisting}
> (or (= 1 2) (> 3 2) (= 2 2))
#t
> (or (> 1 2) (< 3 2) (= 1 4))
#f
> (or (= 3 2) (= 1 1) (thisWouldRaiseAnErrorButDoesNotBecauseItIsNotEvaluated))
#t

\end{lstlisting}


\section*{Builtin Functions}

\subsection*{Arithmetic}

\subsubsection*{add}
\begin{tabular}{l  p{13cm}}
Description: & Adds an arbitrary amount of numbers and returns the accumulated value as \lstinline{SchemeNumber}. If only one argument is given, the arguments value is returned as \lstinline{SchemeNumber}. If no argument is given the return value is 0.\\
Symbol: & \lstinline{+}\\
Arguments: & 0+ \lstinline{SchemeNumber}s\\
Return Value: & \lstinline{SchemeNumber}
\end{tabular}
\\
\\
Example of usage:
\begin{lstlisting}
> (+ 1 2)
3
> (+ 2 3 4)
9
> (+)
0
> (+ 42)
42
\end{lstlisting}

\subsubsection*{subtract}
\begin{tabular}{l  p{13cm}}
Description: & Subtracts an arbitrary amount of numbers from the first number and returns the accumulated value as \lstinline{SchemeNumber}. If only one argument is given, the arguments value is negated and returned as \lstinline{SchemeNumber}. If no argument is given an ArgumentCountException is risen.\\
Symbol: & \lstinline{-}\\
Arguments: & 1+ \lstinline{SchemeNumber}s\\
Return Value: & \lstinline{SchemeNumber}
\end{tabular}
\\
\\
Example of usage:
\begin{lstlisting}
> (- 0.5 2)
-1.5
> (- 10 3 4)
3
> (-)
ArgumentCountException: 'function - expects at least 1 argument.'
> (- 42)
-42
\end{lstlisting}

\subsubsection*{multiply}
\begin{tabular}{l  p{13cm}}
Description: & Multiplies an arbitrary amount of numbers and returns the resulting value as \lstinline{SchemeNumber}. If only one argument is given, the arguments value is returned as \lstinline{SchemeNumber}. If no argument is given the return value is 1.\\
Symbol: & \lstinline{*}\\
Arguments: & 0+ \lstinline{SchemeNumber}s\\
Return Value: & \lstinline{SchemeNumber}
\end{tabular}
\\
\\
Example of usage:
\begin{lstlisting}
> (* 3.5 4)
14.0
> (* 2 3 4)
24
>(*)
1
> (* 42)
42
\end{lstlisting}

\subsubsection*{divide}
\begin{tabular}{l  p{13cm}}
Description: & Divides the first argument by the second, the result by the third and so on. If only one argument is given, the result is 1 devided by the argument. If no argument is given an ArgumentCountException is risen.\\
Symbol: & \lstinline{/}\\
Arguments: & 1+ \lstinline{SchemeNumber}s\\
Return Value: & \lstinline{SchemeNumber}
\end{tabular}
\\
\\
Example of usage:
\begin{lstlisting}
> (- 0.5 2)
-1.5
> (/ 12 3 2)
2.0
> (/)
ArgumentCountException: 'function / expects at least 1 argument.'
> (/ 3)
0.3333333333333333
\end{lstlisting}

\subsubsection*{arithmetic equals}
\begin{tabular}{l  p{13cm}}
Description: & Checks the two arguments for equal value. Returns \lstinline{SchemeTrue} if they are equal, otherwise \lstinline{SchemeFalse}. \\
Symbol: & \lstinline{=}\\
Arguments: & exactly 2 \lstinline{SchemeNumber}s\\
Return Value: & \lstinline{SchemeTrue} or \lstinline{SchemeFalse}
\end{tabular}
\\
\\
Example of usage:
\begin{lstlisting}
> (= 3 3)
#t
> (= 1 2)
#f
> (= 1)
ArgumentCountException: 'function = expects exactly 2 arguments.'
\end{lstlisting}


\subsubsection*{greater than}
\begin{tabular}{l  p{13cm}}
Description: & Returns \lstinline{SchemeTrue} if the first argument is greater than the second one, otherwise \lstinline{SchemeFalse}. \\
Symbol: & \lstinline{>}\\
Arguments: & exactly 2 \lstinline{SchemeNumber}s\\
Return Value: & \lstinline{SchemeTrue} or \lstinline{SchemeFalse}
\end{tabular}
\\
\\
Example of usage:
\begin{lstlisting}
> (> 3 3)
#f
> (> 3 2)
#t
> (> 1)
ArgumentCountException: 'function > expects exactly 2 arguments.'
\end{lstlisting}


\subsubsection*{less than}
\begin{tabular}{l  p{13cm}}
Description: & Returns \lstinline{SchemeTrue} if the first argument is less than the second one, otherwise \lstinline{SchemeFalse}. \\
Symbol: & \lstinline{<}\\
Arguments: & exactly 2 \lstinline{SchemeNumber}s\\
Return Value: & \lstinline{SchemeTrue} or \lstinline{SchemeFalse}
\end{tabular}
\\
\\
Example of usage:
\begin{lstlisting}
> (< 3 3)
#f
> (< 1 2)
#t
> (< 1)
ArgumentCountException: 'function < expects exactly 2 arguments.'
\end{lstlisting}

\subsubsection*{greater or equal}
\begin{tabular}{l  p{13cm}}
Description: & Returns \lstinline{SchemeTrue} if the first argument is greater than or equals the second one, otherwise \lstinline{SchemeFalse}. \\
Symbol: & \lstinline{>=}\\
Arguments: & exactly 2 \lstinline{SchemeNumber}s\\
Return Value: & \lstinline{SchemeTrue} or \lstinline{SchemeFalse}
\end{tabular}
\\
\\
Example of usage:
\begin{lstlisting}
> (>= 3 3)
#t
> (>= 3 2)
#t
> (>= 1 2)
#f
> (>= 1)
ArgumentCountException: 'function >= expects exactly 2 arguments.'
\end{lstlisting}

\subsubsection*{less or equal}
\begin{tabular}{l  p{13cm}}
Description: & Returns \lstinline{SchemeTrue} if the first argument is less than or equals the second one, otherwise \lstinline{SchemeFalse}. \\
Symbol: & \lstinline{<=}\\
Arguments: & exactly 2 \lstinline{SchemeNumber}s\\
Return Value: & \lstinline{SchemeTrue} or \lstinline{SchemeFalse}
\end{tabular}
\\
\\
Example of usage:
\begin{lstlisting}
> (<= 3 3)
#t
> (<= 1 2)
#t
> (<= 3 2)
#f
> (<= 1)
ArgumentCountException: 'function <= expects exactly 2 arguments.'
\end{lstlisting}

\subsubsection*{absolute value}
\begin{tabular}{l  p{13cm}}
Description: & Returns the absolute value of the given argument.\\
Symbol: & \lstinline{abs}\\
Arguments: & exactly 1 \lstinline{SchemeNumber}\\
Return Value: & \lstinline{SchemeNumber}
\end{tabular}
\\
\\
Example of usage:
\begin{lstlisting}
> (abs 3)
3
> (abs -2)
2
> (abs 2 3)
ArgumentCountException: 'function abs expects exactly 1 argument.'
\end{lstlisting}

\subsubsection*{modulo}
\begin{tabular}{l  p{13cm}}
Description: & Does the modulo operation for the two given arguments, i.e. finds the remainder of division of the first argument by the second.\\
Symbol: & \%\\
Arguments: & exactly 2 \lstinline{SchemeNumber}s\\
Return Value: & \lstinline{SchemeNumber}
\end{tabular}
\\
\\
Example of usage:
\begin{lstlisting}
> (% 5 3)
2
> (% 4)
ArgumentCountException: 'function % expects exactly 2 arguments.'
> (% 3.4 1.1)
ArgumentTypeException: '3.4 is no valid operand for procedure %. Expects integer.' 
\end{lstlisting}

\subsection*{Other}

\subsubsection*{exit}
\begin{tabular}{l  p{13cm}}
Description: & Closes the interpreter. Any number of arguments can be given. If the first argument is a SchemeNumber the interpreter will close with the according exit code.\\
Symbol: & \lstinline{cons}\\
Arguments: & exactly two \lstinline{SchemeObject}s\\
Return Value: & \lstinline{SchemeCons}
\end{tabular}
\\
\\
Example of usage:
\begin{lstlisting}
> (exit)
user@computer:~/Studies        

> (exit 12)
user@computer:~/Studies ?12        
\end{lstlisting}

\subsubsection*{print}
\begin{tabular}{l  p{13cm}}
Description: & Prints a representation of the given object to the console.\\
Symbol: & \lstinline{print}\\
Arguments: & \lstinline{SchemeObject}\\
Return Value: & \lstinline{SchemeVoid}
\end{tabular}
\\
\\
Example of usage:
\begin{lstlisting}
> (print "hello")
"hello"
> (print (cons 1 2))
(1 . 2)
> (print (list 4 2 5 6 (+ 1 2)))
(4 2 5 6 3)
> (print nil)
()
\end{lstlisting}

\subsubsection*{display}
\begin{tabular}{l  p{13cm}}
Description: & Nearly similar to print, but prints strings without quotation mark.\\
Symbol: & \lstinline{display}\\
Arguments: & \lstinline{SchemeObject}\\
Return Value: & \lstinline{SchemeVoid}
\end{tabular}
\\
\\
Example of usage:
\begin{lstlisting}
> (display "hello")
hello
> (display (cons 1 2))
(1 . 2)
> (display (list 4 2 5 6 (+ 1 2)))
(4 2 5 6 3)
> (display nil)
()
\end{lstlisting}

\subsubsection*{cons}
\begin{tabular}{l  p{13cm}}
Description: & Creates a \lstinline{SchemeCons} with the first argument as car and the second argument as cdr.\\
Symbol: & \lstinline{cons}\\
Arguments: & exactly two \lstinline{SchemeObject}s\\
Return Value: & \lstinline{SchemeCons}
\end{tabular}
\\
\\
Example of usage:
\begin{lstlisting}
> (cons 1 2)
(1 . 2)
> (cons 1 2 3)
ArgumentCountException: 'cons expects exactly 2 arguments.'  
> (cons (1 (cons 2 3)))
(1 2 . 3)
> (cons 1 (cons 2 nil))
(1 2)
\end{lstlisting}

\subsubsection*{car}
\begin{tabular}{l  p{13cm}}
Description: & Returns the car of the given \lstinline{SchemeCons}.\\
Symbol: & \lstinline{car}\\
Arguments: & \lstinline{SchemeCons}\\
Return Value: & \lstinline{SchemeObject}
\end{tabular}
\\
\\
Example of usage:
\begin{lstlisting}
> (car (cons 1 2))
1
> (car (list "hello" 2 3))
"hello"
> (car 1)
ArgumentTypeException: 'car expects cons as argument'   
\end{lstlisting}

\subsubsection*{cdr}
\begin{tabular}{l  p{13cm}}
Description: & Returns the cdr of the given \lstinline{SchemeCons}.\\
Symbol: & \lstinline{cdr}\\
Arguments: & \lstinline{SchemeCons}\\
Return Value: & \lstinline{SchemeObject}
\end{tabular}
\\
\\
Example of usage:
\begin{lstlisting}
> (cdr (cons 1 2))
2
> (car (list "hello" 2 3))
(2 3)
> (cdr 1)
ArgumentTypeException: 'cdr expects cons as argument'   
\end{lstlisting}


\subsubsection*{list}
\begin{tabular}{l  p{13cm}}
Description: & Creates a regular list out of all arguments.\\
Symbol: & \lstinline{list}\\
Arguments: & 0+ \lstinline{SchemeObject}s\\
Return Value: & \lstinline{SchemeCons} or \lstinline{SchemeNil}
\end{tabular}
\\
\\
Example of usage:
\begin{lstlisting}
> (list 1 2 3)
(1 2 3)
> (list 1)
(1)
> (list)
()
\end{lstlisting}


\subsubsection*{list?}
\begin{tabular}{l  p{13cm}}
Description: & Returns \lstinline{SchemeTrue} if the argument is a regular list, else \lstinline{SchemeFalse}.\\
Symbol: & \lstinline{list?}\\
Arguments: & \lstinline{SchemeObject}\\
Return Value: & \lstinline{SchemeTrue} or \lstinline{SchemeFalse}
\end{tabular}
\\
\\
Example of usage:
\begin{lstlisting}
> (list? (list 1 2 3))
#t
> (list? (cons 1 2))
#f
> (list? 1)
#f
> (list? nil)
#t
\end{lstlisting}


\subsubsection*{first}
\begin{tabular}{l  p{13cm}}
Description: & Returns the first element of the given list. A regular list is expected.\\
Symbol: & \lstinline{first}\\
Arguments: & \lstinline{SchemeCons} - has to be a regular list.\\
Return Value: & \lstinline{SchemeObject}
\end{tabular}
\\
\\
Example of usage:
\begin{lstlisting}
> (first (list 1 2 3))
1
> (first (list "hello" "world"))
"hello"
> (first (cons "hello" "world"))
ArgumentTypeException: 'rest expects a not empty list as argument.' 
\end{lstlisting}

\subsubsection*{rest}

\begin{tabular}{l  p{13cm}}
Description: & Returns the rest list after the first argument of the given list. A regular list is expected.\\
Symbol: & \lstinline{rest}\\
Arguments: & \lstinline{SchemeCons} - has to be a regular list.\\
Return Value: & \lstinline{SchemeObject}
\end{tabular}
\\
\\
Example of usage:
\begin{lstlisting}
> (rest (list 1 2 3))
(2 3)
> (rest (cons 1 2))
ArgumentTypeException: 'rest expects a not empty list as argument.'  
> (rest (list "hello" "world"))
("world")
\end{lstlisting}



\subsubsection*{time}

\subsubsection*{recursion-limit}
\begin{tabular}{l  p{13cm}}
Description: & If no argument is given the current recursion limit is returned. Per default this is 1000. If a \lstinline{SchemeNumber} is given, the recursion limit is set to this number.\\
Symbol: & \lstinline{recursion-limit}\\
Arguments: & nothing or \lstinline{SchemeNumber}\\
Return Value: & \lstinline{SchemeVoid} or \lstinline{SchemeNumber}
\end{tabular}
\\
\\
Example of usage:
\begin{lstlisting}
> (recursion-limit)
1000
> (recursion-limit 2000)
> (recursion-limit)
2000
\end{lstlisting}

\subsubsection*{type?}
\begin{tabular}{l  p{13cm}}
Description: & Evaluates the given \lstinline{SchemeObject} and returns the type of the return value as \lstinline{SchemeString}.\\
Symbol: & \lstinline{type?}\\
Arguments: & \lstinline{SchemeObject}\\
Return Value: & \lstinline{SchemeString}
\end{tabular}
\\
\\
Example of usage:
\begin{lstlisting}
> (type? 1)
"schemeNumber"
> (type? (define a 1))
"schemeVoid"
> (type? (quote (+ 1 2)))
"schemeCons"
\end{lstlisting}

\subsubsection*{not}
\begin{tabular}{l  p{13cm}}
Description: & Returns \lstinline{SchemeTrue} for \lstinline{SchemeFalse}, \lstinline{SchemeFalse} for everything else.\\
Symbol: & \lstinline{not}\\
Arguments: & \lstinline{SchemeObject}\\
Return Value: & \lstinline{SchemeTrue} or \lstinline{SchemeFalse}
\end{tabular}
\\
\\
Example of usage:
\begin{lstlisting}
> (not #t)
#f
> (not #f)
#t
> (not 1)
#f
> (not (list 1 2 3))
#f
> (not true)
#f
> (not false)
#t
\end{lstlisting}


\subsubsection*{map}
\begin{tabular}{l  p{13cm}}
Description: & Executes the given function (first argument) for every element of the given list (second argument) and returns a list out of all results.\\
Symbol: & \lstinline{map}\\
Arguments: & \lstinline{SchemeUserDefinedFunction} or \lstinline{SchemeBuiltinFunction} and \lstinline{SchemeCons} (has to be a list)\\
Return Value: & \lstinline{SchemeCons}
\end{tabular}
\\
\\
Example of usage:
\begin{lstlisting}
> (define (add1 n) (+ n 1))
> (define l (list 1 2 3 4))
> (map add1 l)
(2 3 4 5)
\end{lstlisting}

\subsubsection*{get-function-info}
\begin{tabular}{l p{13cm}}
Description: & Prints a representation of the given user defined function to the console. It includes the name, the parameters and the function body.\\
Symbol: & \lstinline{get-function-info}\\
Arguments: & \lstinline{SchemeUserDefinedFunction}\\
Return Value: & \lstinline{SchemeVoid}
\end{tabular}
\\
\\
Example of usage:
\begin{lstlisting}
> (define (f n m) (+ n 1) (- n m) (+ n m))
> (get-function-info f)
name: f
arglist:
	n
	m
bodylist:
	(+ n 1)
	(- n m)
	(+ n m)
\end{lstlisting}





\end{document}